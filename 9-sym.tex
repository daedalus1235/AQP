%! TEX root = 0-main.tex
\chapter{Symmetries in 2-Body Problems}
These next two chapters consider the quantum mechanics of central potentials. Recall the form of the laplaccian in spherical coordinates:
\[\del^2 = \frac{1}{r^2}\pder{}{r}r^2\pder{}{r} + \frac{1}{r^2\sin\theta}\pder{}{\theta}\sin\theta\pder{}{\theta} + \frac{1}{r^2\sin^2\theta}\pder{^2}{\phi^2}\]
Often, we assume that the wavefunction is separable into a radial and angular part:
\[\P = R(r)Y(\theta,\phi)\]
We will take these ``brute-force'' methods later; there is often a more elegant method to find functional forms for the solutions by abusing symmetries to construct operators and corresponding eigenfunctions. If the symmetry operator commutes with the Hamiltonian, then we see that simultaneous diagonalization applies, so we can use our previously determined symmetry eigenfunctions to construct the stationary states. Interestingly, these stationary states correspond to conservation laws which give conservation laws

\section{Symmetries}
Recall the case of the 1D free particle. In the absence of a potential, space is homogeneous, so we defined the translation operator. 
\[\hat T(a)\k{x} = \k{x+a}\]
We determined that the generator of translation was momentum; this translational symmetry corresponds to the conservation of linear momentum.

Similarly, recall the 1D harmonic oscillator. The potential was an even function, so we defined the parity operator. Correspondingly, we found that solutions had to have definite parity (even, odd).

\subsection{Translation in 3D}
We generalize our previous discussion of translation to higher dimensions. The position ket is expanded such that
\[\k{\vv r} = \k{x,y,z}\]
with the property that
\[\bk{\vv r'}{\vv r}= \delta(x'-x)\delta(y'-y)\delta(z'-z) = \delta^3(\vv r' - \vv r)\]
Correspondingly, we define a translation operator
\[\hat T(\vv a)\k{\vv r} = \k{\vv r +\vv a}\]
We find that the translation operator is as expected; it is generated by the momemtum;
\begin{equation}
	\hat T(\vv a) = \exp[-i\hat{\vv p} * \vv a /\hbar]
\end{equation}
We can similarly find the position-space representation of the momentum as
\begin{equation}
	\b{\vv r} \hat{\vv p}\k\p = \frac{\hbar}{i}\del\p(\vv r)
\end{equation}
We again have
\begin{equation}
	\bk{\vv r}{\vv p} = \frac{i}{(2\pi\hbar)^{3/2}} e^{i\vv p*\vv r/\hbar}
\end{equation}
and the commutation relation
\begin{equation}
	[\hat x_i,\hat p_j] = i\hbar \delta_{ij}
\end{equation}
The proofs for these are analogous to those done in the initial treatment of 1D translation.

\section{Two-Body Problem}
Consider a Hamiltonian of distinguishable\footnote{We will encounter \emph{indistinguishable particles} later.}
\begin{equation}
	\hat H = \frac{\hat p_1^2}{2m_1}+\frac{\hat p_2^2}{m_2} + V\left(\norm{\hat{\vv r}_1-\hat{\vv r}_2}\right)
\end{equation}
Neglecting spin, we have the position states
\[\k{\vv r_1, \vv r_2} = \k{\vv r_1}_1\tp \k{\vv r_2}_2\]
We then consider the operator that translates both particles
\[\hat T_1(\vv a)\tp\hat T_2(\vv a) = e^{-i\hat{\vv P}*\vv a/\hbar}\]
where \(\hat{\vv P}=\hat{\vv p}_1 + \hat{\vv p}_2\). We expect, as the system is invariant under translation (there is no external potential) that this total momentum commutes with the Hamiltonian. Clearly, we have
\[[\hat p_{ni},\hat p_{ni}^2]=0\]
for example, \([\hat p_{1x}, \hat p_{1x}^2]\). Similarly, by using the commutivity of translation operators (in particular, over infinitessimal distances), we can show that
\[[\hat p_{ni},\hat p_{nj}]=0\]
Thus, we can show that
\[[\hat{\vv P},\hat{\vv p_n}]=0\]
Finally, we can show that the momentum commutes with the potential. This will be done in the homework.
Using this fact, we wil consider the new quantities relative position
\[\hat{\vv r} = \hat{\vv r}_1 - \hat{\vv r}_2\]
CoM,
\[\hat{\vv R} = \frac{m_1\hat{\vv r}_1+m_2\hat{\vv r}_2}{m_1+m_2}\]
and relative momentum
\[\hat{\vv p} = \frac{m_2 \hat{\vv p}_1-m_1\hat{\vv p}_2}{m_1+m_2}\]
allowing us to rewrite the Hamiltonian as
\begin{equation}
	\hat H = \frac{\hat P^2}{2M} +\frac{\hat p^2}{2\mu}+V(\norm{\hat{\vv r}})
\end{equation}
where of course, \(M=m_1+m_2\) is the total mass and \(\mu = \frac{m_1m_2}{m_1+m_2}\). We can write the energy eigenstates in terms of \(\k{\vv R,\vv r}\). Clearly only the \(\hat P^2\) term impacts \(\hat R\). This lets us rewrite the hamiltonian as
\begin{equation}
	\hat H_{\text{tot}} = \frac{\hat P^2}{2m}+\hat H_{\text{rel}}
\end{equation}
Thus, we have the total motion of the system given
\[\bk{\vv R}{\vv P} = \frac{1}{(2\pi\hbar)^{3/2}}e^{i\vv P*\vv R/\hbar}\]
This reduces the problem to that of the relative motion:
\begin{equation}
	\hat H_{\text{rel}} = \frac{\hat p^2}{2\mu}+V(\hat r)
\end{equation}

\section{Rotational Symmetry}
To examine rotational symmetry, we define a rotation operator. As before,
\[\hat R(\d\phi \hat k)\k{x,y,z} = \k{x-y\d\phi, y+x\d\phi, z}\]
We can write this operator in terms of a generator, the orbital angular momentum:
\begin{equation}
	\hat R(\d\phi\kh) = \mathbbm 1 - \frac{i}{\hbar}\hat L_z \d\phi
\end{equation}
We expect, due to the central potential, that
\[[\hat R,\hat H] = 0\]
To do that, we need only consider
\[[\hat L, \hat H]= 0\]
returning to the definition of the rotation of an infinitessimal angle, we see we can write the transformation in terms of two translations:
\begin{align*}
	\mathbbm 1 - \frac{i}{\hbar}\hat L_z \d\phi &= \left[\mathbbm 1 -\frac{i}{\hbar}\hat p_x(-y\d\phi)\right]\left[\mathbbm 1 - \frac{i}{\hbar}\hat p_y(x\d\phi)\right]\\\intertext{Keeping only first order contributions in \(\d\phi\),}
						    &=\mathbbm 1 - \frac{i}{\hbar}\left(\hat x \hat p_y - \hat y\hat p_x\right)\d\phi
\end{align*}
We can show similar relations for the other coordinates to obtain
\begin{equation}
	\hat{\vv L} = \hat{\vv r}\times \hat{\vv p}
\end{equation}
which matches the classical result. Examining the commutator for the \(z\) component of orbital angular momentum,
\[[\hat L_z, \hat p_z] = [\hat x\hat p_y-\hat y\hat p_x, \hat p_z] = 0\]
\[[\hat L_z, \hat p_x] = \hat x\cancelto{0}{[p_y, p_x]}+[\hat x, \hat p_x]\hat p_y = i\hbar\hat p_y\]
similarly, 
\[[\hat L_z, \hat p_y] = -i\hbar \hat p_x\]
Thus, it is routine to show that
\[[\hat L_z, \hat p^2] = 0\]
We can show the same holds true for the magnitude of position:
\[[\hat L_z,\hat z] = 0 \qquad\qquad [\hat L_z, \hat x] = i\hbar \hat y \qquad\qquad [\hat L_z, \hat y] = -i\hbar \hat x\]
Note that all of the previously discussed commutation relations are very similar to ``cross products'' when considering directions. Thus, we have
\[[\hat L_z, \hat r^2]=0\]
We can show the same holds true for the the other components of orbital angular momentum; thus we can see that the components of orbital angular momentum commute with the hamiltonian, so we see trivially that
\begin{equation}
	[\hat L^2, \hat H] = 0
\end{equation}
Finally, in analogy to the spin angular momentum, we have the result
\begin{equation}
	[\hat L^2, \hat L_z] = 0
\end{equation}
Thus, we can form simultaneous eigenstates of \(\hat H\) with one component of orbital angular momentum, and \(\hat L^2\). We label these eigenvalues with \(E, m\hbar, \ell(\ell+1)\hbar^2\) respectively:
\begin{subequations}
	\begin{align}
		\hat H\k{E,\ell,m}& = E\k{E,\ell,m}\\
		\hat L^2\k{E,\ell,m}& = \ell(\ell+1)\hbar^2\k{E,\ell,m}\\
		\hat L_z\k{E,\ell,m}& = m\hbar\k{E,\ell,m}
	\end{align}
\end{subequations}
\section{Eigenfunctions}
\subsection{\texorpdfstring{\(\hat L_z\)}{Lz} Eigenfunction}
We have the representation of the \(\hat L_z\) operator as
\[\hat L_z = \hat x \hat p_y - \hat y \hat p_x = \frac{\hbar}{i}\left(x\pder{}{y}-y\pder{}{x}\right)\]
We can simplify this using spherical coordinates, to
\[\hat L_z = \frac{\hbar}{i}\pder{}{\phi}\]
Thus, the eigenfunctions are given:
\[\frac{\hbar}{i}\pder{\Phi}{\phi} = m\hbar \Phi\]
so, we obtain the eigenfunctions
\begin{equation}
	\Phi = e^{im\phi}\qquad m\in\Z
\end{equation}
Note, that orbital angular momentum can only have full-integer values, as opposed to the half-integer values that spin angular momentumm can take.

\subsection{\texorpdfstring{\(\hat L^2\)}{L2} Eigenfunctions}
We can similarly express the \(\hat L^2\) operator in spherical coordinates as
\begin{equation}
	\hat L^2 = -\hbar^2 \left[\frac{1}{\sin\theta}\pder{}{\theta}\sin\theta\pder{}{\theta}+\frac{1}{\sin^2\theta}\pder{^2}{\phi^2}\right]
\end{equation}
Note that this is the angular part of the operator \(r^2\del^2\) as expressed in spherical coordinates!

\begin{aside}[Spherical Hamiltonian]
	We can write the hamiltonian in spherical coordinates as
	\[H\simeq -\frac{\hbar^2}{2\mu}\left(\pder{}{r}+\frac{1}{r}\right)^2 + \frac{\hat L^2}{2\mu r^2}+V(r)\]
	We recognize the term
	\[\hat p_r \simeq \frac{\hbar}{i} \left(\pder{}{r}+\frac{1}{r}\right)\]
	The slightly unintuiitive form can be derived from the classical form using
	\[p_r = \frac{1}{2}(\vv p * \vv r + \hat r * \vv p)\]
	Thus, we obtain
	\[\hat H = \frac{\hat p_r^2}{2\mu} + \frac{\vv L^2}{2\mu \hat r^2}+V(\hat r)\]
\end{aside}

From the representation of the \(\hat L^2\) operator in spherical coordinates, we see that the eigenfunctions depend solely on angular components. We denote
\[\bk{\theta, \phi}{\ell, m} = Y_{\ell m}(\theta, \phi)\]
We can consider this eigenvalue problem to be that of a particle restricted to the surface of a sphere, the manifold \(S^2\). Thus, we obtain the eigenfunctions of \(\hat H\) to be
\[\bk{r,\theta,\phi}{E,\ell, m } = R(r)Y_{\ell m}(\theta,\phi)\]
We can solve for the eigenfucntion \(Y_{\ell m}\) by defining a differential equation from the eigenvalue problem
\[\hat L^2 Y_{\ell m} = \ell(\ell+1) Y_{\ell m}\]
This solution requires the laborious use of a power series solution. However, a more elegant solution is to use the rasing and lowering operators. 
\subsubsection{Raising and Lowering Operators}
Recall from our discussion of spin that we have 
\begin{equation}
	\hat L_\pm = \hat L_x\pm i\hat L_y
\end{equation}
The representation of the \(\hat L_x\) and \(\hat L_y\) operators in spherical coodinates can be given
\begin{subequations}
\begin{align}
	\hat L_x&= \frac{\hbar}{i}\left(-\sin\phi\pder{}{\theta} - \cot\theta\cos\phi\pder{}{\phi}\right)\\
	\hat L_x&= \frac{\hbar}{i}\left(-\sin\phi\pder{}{\theta} - \cot\theta\cos\phi\pder{}{\phi}\right)
\end{align}	
\end{subequations}
Thus, we obtain the representations
\begin{equation}
	\hat L_\pm = \frac{\hbar}{i}e^{\pm i\phi}\left(\pm i \pder{}{\theta}-\cot\theta\pder{}{\phi}\right)
\end{equation}
Pulling out the \(\phi\) dependence from the \(\hat L_z\) operator, we now have a new expression
\[Y_{\ell m} = e^{im\phi}f(\theta)\]
Applying this to the equation
\[\hat L_+ \k{\ell,\ell} = 0\]
we obtain (after simplification) the easier differential equation
\[\left(\der{}{\theta} - \ell\cot\theta\right)f(\theta) = 0\]
This yields the solution
\[f(\theta) = c_\ell \sin^\ell \theta\]
so we obtain
\begin{equation}
Y_{\ell\ell} = c_\ell e^{i\ell\phi}\sin^\ell\theta
\end{equation}
We want this expression to be normalized.
\[\int\d[3]r\abs{R(r)}^2\abs{Y_{\ell m}(\theta,\phi)}^2 = 1\]
This is separable:
\[\int r^2\d{r} \abs{R(r)}^2\int\d{\Omega}\abs{Y_{\ell m}(\theta,\phi)}^2=1\]
For convenience, we define the radial and angular eigenfunctions to be normalized independently.
\begin{aside}[Solid Angle Integral]
	We have the following equivalent expressions for the integration over the surface of a sphere:
	\[\int_0^\pi\sin\theta\d\theta \int_0^{2\pi}\d\phi = \int_{-1}^1\d{(\cos\theta)} \int_0^{2\pi}\d\phi = \int\d\Omega\]
\end{aside}
This yields the the normalization factor
\[c_\ell = \frac{(-1)^\ell}{2^\ell\ell!}\sqrt{\frac{(2\ell+1)!}{4\pi}}\]
Note that this coefficient has a phase factor that is dependent on \(\ell\). This is doesn't affect the normalization, but turns out to be convenient.

To obtain the eigenfunctions where \(m\neq \ell\) we use the fact that
\begin{equation}
	\hat L_-\k{\ell, m} = \hbar\sqrt{\ell(\ell+1)-m(m-1)}\k{\ell, m-1}
\end{equation}
and repeatedly apply the lowering operator. Thus, we can write
\begin{equation}
Y_{\ell m} = \frac{(-1)^\ell}{2^\ell \ell!}\sqrt{\frac{2\ell+1}{4\pi}*\frac{(\ell+m)!}{(\ell-m)}!} e^{im\phi}\frac{1}{\sin^m\theta}\frac{\d[\ell-m]{}}{\d{(\cos\theta)^{\ell-m}}}\sin^{2\ell}\theta
\end{equation}
for \(m\geq 0\). The \((-1)^\ell\) ensures a positive value for \(\theta=0\). For \(-m<0\), we have
\begin{equation}
	Y_{\ell, -m} = (-1)^{m}Y_{\ell, m}\ast
\end{equation}
Interestingly, we have
\[Y_{\ell, 0}(\theta, \phi) = \sqrt{\frac{2\ell+1}{4\pi}}P_{\ell}(\cos\theta)\]
where \(P_\ell\) is the \(\ell\)\textsuperscript{th} \emph{Legendre polynomial}. The functions \(Y_{\ell, m}(\theta,\phi)\) are known as the \emph{spherical harmonics}.

We identify \(Y_{0,0}\) as an s state, and \(Y_{1,m}\) produces a p state. Similarly, the \(Y_{2,m}\) states correspond to d orbitals.

In fact, we can take linear combinations of the \(Y_{\ell m}\) orbitals to obtain real wavefunctions. The \(\ell = 0\) state corresponds directly to the \(s\) orbital, and is spherically symmetric. The \(Y_{1,0} = \sqrt{\frac{3}{4\pi}}\frac{z}{r} = p_z\), \(\frac{Y_{1,-1}-Y_{1,1}}{\sqrt{2}} = \sqrt{\frac{3}{4\pi}}\frac{x}{r}=p_x\) and \(\frac{i(Y_{1,1}+Y_{1,-1})}{\sqrt{2}} = \sqrt{\frac{3}{4\pi}}\frac{y}{r} = p_y\) now yield definite phases in each of the lobes. Similar linear combinations can be made from \(Y_{2,m}\) to form the d orbitals, \(Y_{3,m}\) to form the \(f\) orbitals, and so forth.

\subsection{Spectroscopy}
\subsubsection{Rigid Rotor}
Consider a molecule of two atoms \(m_1, m_2\) with a potential with a characteristic bond length \(r_0\). The hamiltonian can then be written. Neglecting the potential, and considering it in the CoM frame, the hamiltonian is given
\[\hat H = \frac{\hat L^2}{2\mu_N r_0^2} = \frac{\hat L^2}{2I}\]
where \(\mu_N\) is the \emph{reduced nucleon mass}, and is the reduced mass of the two nuclei. As such, we obtain
\[E = \frac{\ell(\ell+1)\hbar^2}{2\mu_Nr_0^2}\simeq \frac{\hbar^2c^2}{2\mu_nc^2a_0^2}\ell(\ell+1) \sim\frac{1}{\tilde n}\frac{m_e}{m_N}\frac{m_ee^4}{\hbar^2}\simeq\SI{1}{meV} \]
where, of course \(a_0 = \frac{\hbar^2}{m_ee^2}\approx \SI{0.529}{\angstrom}\) is the Bohr radius, \(m_e/m_N\sim 2000\), and \(\frac{m_e e^4}{2\hbar^2} = \SI{1}{Ryd} \sim \SI{13.6}{eV}\)
This give spectral separation at about microwaves.

\subsubsection{Vibration}
Consider small oscillations about the equilibrium, and approximate as a harmonic oscillator with
\[\der{V}{r2} \approx \frac{e^2/a}{a^2}\then m\omega^2 = e^2/a^3\]
Thus, we have
\[\hbar\omega=\hbar\left(\frac{1}{\tilde n_N}*\frac{1}{M_N}\frac{e^2}{a^3}\right)^{1/2}\simeq \frac{1}{\sqrt{\tilde n_N}}\left(\frac{m_e}{M_N}\right)^{1/2}\left(\frac{m_ee^4}{\hbar^2}\right) \simeq \SI{0.5}{eV}\]
This is in the NIR range.

\subsubsection{Absorption vs Emission}
For emission, we often use photomultiplier tubes. However, the incident light to the PMT must exceed the work function of the cathode; the emission from the rigid rotor and vibration is nowhere near energetic enough to be detected. As such, vibrotational spectra are typically measured using absorption spectra. 

In emission, it is possible to obtain \emph{sidebands}. For most molecules, the HOMO-LUMO gap is on the order of \si{eV}. The decay of electronic exications will often decay into excited vibrational modes, resulting in so-called \emph{phonon sidebands}. The separation between these phonon sidebands allows a sensitive method to measure the vibrational modes.

\subsubsection{Ring of Carbons}
Consider a cyclic potential, such as the pi orbitals in a benzene. This is similar to a constrained rigid rotor. The hamiltonian becomes:
\[\hat H = \frac{\hat L_z^3}{2m_e r_0^2}\]
Thus, trivially, we have the energy levels as
\[E = \frac{m^2\hbar^2}{2m_er_0^2}\]

\section{Separation of Variables}
We can solve the Schr\"odinger equation instead by a power series solution and separation of variables. We can consider 
\[\Psi = RY\]
Acting on Schrodinger equation, and dividng through by \(\Psi/r^2\), we obtain two equations:
\[\frac{1}{R}\left(\pder{}{r}r^2\pder{R}{r}\right) + \frac{2\mu r^2}{\hbar^2}\left(E-V(r)\right) = -\frac{1}{Y}\left[\frac{1}{\sin\theta}\pder{}{\theta}\sin\theta\pder{Y}{\theta} + \frac{1}{\sin^2\theta}\pder{^2Y}{\phi^2}\right]\]
Becuse the LHS is a function of only \(r\), while the RHS is a function of only \(\theta,\phi\), we must have that bot sides are equal to a constant, \(\lambda\). We then have two differential equations. The LHS becomes
\[\frac{1}{r^2}\der{}{r}\left(r^2\der{R}{r}\right) + \left[\frac{2\mu}{\hbar^2}(E-V(r))-\frac{\lambda}{r^2}\right]R = 0\]
The term \(\lambda/r^2\) is the \emph{centrifugal barrier}. The RHS becomes
\[\frac{1}{\sin\theta}\pder{}{\theta}\left(\sin\theta\pder{Y}{\theta}\right)+\frac{1}{\sin^2\theta}\pder{^2Y}{\phi}+\lambda Y=0\]
We once again ``assume'' this equation is separable into \(Y = \Theta\Phi\), so we obtain, for a new constant \(\nu\),
\[\der{\Phi}{\phi2} + \nu\Phi = 0\]
\[\frac{1}{\sin\theta}\der{}{\theta}\left(\sin\theta\der{\Theta}{\theta}\right)+\left(\lambda - \frac{\nu}{\sin^2\theta}\right)\Theta = 0\]
Trivially, we find that \(\nu = m^2, m\in \Z\), so 
\[\Phi = \frac{1}{\sqrt{2\pi}}e^{im\phi}\]
Plugging this into the equation for \(\Theta\), and transforming \(\Theta(\theta) = P(u=\cos\theta)\), we obtain
\[\der{}{u}\left[(1-u^2)\der{P}{u}\right]+\left(\lambda - \frac{m^2}{1-u^2}\right)P = 0\]
Setting \(m=0\), we obtain Legendre's equation. We can try a power series solution \(P = \sum_k a_k u^k\). We obtain the recursion relation for coefficients as
\begin{equation}
	\frac{a_{k+2}}{a_k} =\frac{k(k+1)-\lambda}{(k+2)(k+1)}
\end{equation}
Note, that this tends to \(1-\frac{1}{n}\) as \(k\to\infty\), and so the infinite series diverges.\footnote{If the ratio between coefficients can be written as \(1-\frac{s}{n}\), the series converges iff \(s>1\)} Thus, we must have, for some \(k\), that 
\[k(k+1)-\lambda_\ell = 0\]
so, 
\[\ell(\ell+1) = k(k+1) = \lambda_\ell\]


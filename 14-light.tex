%! TEX root = 0-main.tex
\chapter{Absorption and Emission of Light}
We will now diverge from Townsend. Rather than considering a quantized EM field, we will rather use a classical treatment of EM waves. This is because the discussion of a quantized EM field is a very complex topic and discussion will take too long. Indeed, most undergraduate courses only consider this semi-classical perturbation theory, leaving the quantized field to graduate courses.

\section{Time-dependent Perturbation Theory}
Recall that the time evolution of a state can be given
\[\k{\p(t)} = e^{-i Ht/\hbar}\k{\p(0)} = \sum_n\k{E_n}\bk{E_n}{\p(0)} e^{-iE_n t/\hbar}\]
However, as with time-independent perturbations, we often cannot solve for the eigenstates directly. Thus, we will write the hamiltonian in terms of a known hamiltonian and a perturbing hamiltonian, as we did before. We will assume the time-dependent wave function can be stated as
\[\k{\p(t)} = \sum_n c_n(t) e^{-E_n^{(0)}t/\hbar}\k{E_n^{(0)}}\]
It is these functions \(c_n\) that we will try to obtain perturbatively. Plugging into the time-dependent schr\"odinger equation, we obtain
\[H\k{\p(t)} = i\hbar \der{}{t}\k{\p(t)}\]
\[\sum_n c_n(t) e^{-iE_n^{(0)}t/\hbar}(H_0+H_1)\k{E_n^{(0)}} = i\hbar\sum_n\left[\der{c_n}{t} - \frac{ie_n^{(0)}}{\hbar}c_n\right]e^{-iE_n^{(0)}t/\hbar}\k{E_n^{(0)}}\]
Taking the inner product of both sides with \(\b{E_f^{(0)}}\) and rearranging, we obtain
\begin{equation}
	\der{c_n}{t} = -\frac{i}{\hbar}\sum_{n}c_n e^{i(E_f^{(0)}-E_n^{(0)})t/\hbar}\b{E_f^{(0)}}H_1 \k{E_n^{(0)}}\label{eq14:ctime}
\end{equation}
which yields an infinite system of coupled linear differential equations. Assume that at \(t=0\) that the system is in the state \(\k{E_i^{(0)}}\), or that \(c_n(0) = \delta_{ni}\). Writing 
\[H = H_0+\lambda H_1\]
we can match coefficients to obtain
\[c_n(t) = c_n^{(0)}+\lambda c_n^{(1)} + \lambda^2c_n^{(2)}+\dots\]
as we did before in the time-independent case. Plugging these expressions into Eq.~\ref{eq14:ctime}, 
\[\der{}{t}\sum_j \lambda^j c_f^{(j)} = -\frac{i}{\hbar}\sum_n\left(\sum_j\lambda^j c_n^{(j)}\right) e^{i(E_f^{(0)}-E_n^{(0)})t/\hbar}\b{E_f^{(0)}}\lambda H_1\k{E_n^{(0)}}\]
Collecting terms, we see for \(\lambda^0\), we see 
\begin{equation}
	\der{}{t}c_f^{(0)} = 0\label{eq14:c0diff}
\end{equation}
Thus, our initial condition shows that \(c_f^{(0)} = \delta_{fi}\) and \(c_n^{(k\geq1)}(t=0) = 0\). Our first order correction, \(\lambda^1\), shows us
\[\der{}{t}c_f^{(1)} = -\frac{i}{\hbar}\sum_n c_n^{(0)} e^{i(E_f^{(0)}-E_n^{(0)})t/\hbar}\b{E_f^{(0)}}H_1\k{E_n^{(0)}}\]
Substituting our initial condition \(c_n^{(0)} = \delta_{ni}\), this reduces to
\begin{equation}
	\der{}{t}c_f^{(1)} = -\frac{i}{\hbar}e^{i(E_f^{(0)}-E_i^{(0)})t/\hbar}\b{E_f^{(0)}}H_1\k{E_i^{(0)}}\label{eq14:c1diff}
\end{equation}
or, integrating,
\begin{equation}
	c_f^{(1)}(t) = \int_0^t\d{t'} e^{i(E_f^{(0)}-E_i^{(0)})t'/\hbar}\b{E_f^{(0)}}H_1\k{E_i^{(0)}}\label{eq14:c1int}
\end{equation}
Defining \(\omega_{fi} = \frac{E_f^{(0)}-E_i^{(0)}}{\hbar}\), our first order approximation becomes
\begin{equation}
	c_f(t) = \delta_{fi}-\frac{i}{\hbar}\int_0^t\d{t'}e^{i\omega_{fi}t'}\b{E_f^{(0)}}H_1\k{E_i^{(0)}}\label{eq14:foa}
\end{equation}

\subsubsection{Light}
In our application to the absorption and emission of light, we will have 
\[H_1\propto e^{-i\omega t}\]
so 
\[\b{E_f^{(0)}} H_1 \k{E_i^{(0)}}\propto e^{-i\omega t}\]
Pluggin into our first order approximation, 
\[c_f(t) = \delta_{fi} -\frac i\hbar\int_0^t \d{t'} A e^{i(\omega_{fi}-\omega)t}\]
This integral is appreciable only for \(\omega=\omega_{fi}\), which forces the exponeent to unity.

Again, because we are only considering the semi-classical approach, there are no photons. Instead, we expect to observe oscillations between those two states, and energy moves back and forth between the system and the electric field.

\subsection{1D Harmonic Oscillator in a Decaying Electric Field}
Consider a harmonic oscillator in the field \(\mathcal E = \mathcal E e^{t/\tau}\), and so our hamiltonian becomes
\[H = \frac{p^2}{2m} +\frac{1}{2}m\omega x^2 - q\mathcal E x e^{-t/\tau}\]
Writing \(x\) in terms of the raising and lowering operator, we see
\[H_1 = q\mathcal E e^{-t/\tau}\sqrt{\frac{\hbar}{2m\omega}}(a+a\adj)\]
Thus, we can find the probability of the operator going from the ground state to an excited state by using
\begin{align*}
	c_n(\infty) &= \frac{iq\mathcal E}{\hbar}\sqrt{\hbar}{2m\omega}\int_0^\infty\d{t'} e^{in\omega t'}e^{-t'/\tau}\b{n} a+a\adj\k{0}\\
	&=\delta_{n1}\frac{iq\mathcal E}{\hbar}\sqrt{\frac{\hbar}{2m\omega}}\int_0^\infty\d{t'} e^{i\omega t'}e^{-t'/\tau}\\
	&=\delta_{n1} \frac{iq\mathcal E}{\hbar}\sqrt{\frac{\hbar}{2m\omega}}\frac{\tau}{1-i\omega\tau}\\
	P = \abs{c_1(\infty)}^2  &= \frac{(q\mathcal E \tau)^2}{2m\hbar\omega}\frac{1}{1+\omega^2 \tau^2}
\end{align*}
Thus, to first order, only the \(\k{0}\to \k{1}\) transition occurs. More generally, if the state starts as \(\k{n}\) we can only have transitions to \(\k{n\pm 1}\), or \(\Delta n\). This is an example of a \emph{selection rule} (though this one is dependent on the specific shape of the well).

\subsection{Sinusoidal Perturbation}
Consider a perturbation
\[H_1 = V\cos\omega t\]
Defining the matrix element \(V_{fi} = \b{E_f^{(0)}}V\k{E_i^{(0)}}\), and expanding out the cosine, we can find the coefficient as
\begin{align*}
	c_f(t) &= -\frac{iV_{fi}}{2\hbar}\int_0^t \d{t'} e^{i(\omega_{fi}-\omega)t'}+e^{i(\omega_{fi}-\omega)t'}\\
	       &=-\frac{V_{fi}}{2\hbar}\left(\frac{e^{i(\omega_{fi}+\omega)t}-1}{\omega_{fi}+\omega} + \frac{e^{i(\omega_{fi}-\omega)t}-1}{\omega_{fi}-\omega}\right)
\end{align*}
The term on the right dominates when \(\omega\sim\omega_{fi}\), and corresponds with a positive \(\omega_{fi}\) (where we fix \(\omega>0\) WLOG), so \(\Delta E = \hbar \omega_{fi}>0\) and corresponds to \emph{absorption}. On the other hand, the first term dominates when \(\omega\sim -\omega_{fi}\), has \(\Delta E<0\), and corresponds to \emph{stimulated emission}.

More generally, we can pull out a term to write
\begin{align*}
	c_f(t) &= -\frac{V_{fi}}{2\hbar} \frac{e^{i(\omega_{fi}\pm \omega)t/2}}{\omega_{fi}\pm\omega}\left(e^{i(\omega_{fi}\pm \omega)t/2}-e^{-i(\omega_{fi}\pm\omega)t/2}\right)\\
	       &=-\frac{iV_{fi}}{\hbar}\frac{\sin[(\omega_{fi}\pm)t/2]}{\omega_{fi}\pm\omega}e^{i(\omega_{fi}\pm \omega)t/2}\\
	P_{i\to f} = \abs{c_f(t)}^2&=\frac{\abs{V_{fi}}^2}{\hbar^2}\frac{\sin^2[(\omega_{fi}\pm\omega)t/2]}{(\omega_{fi}\pm\omega)^2}\\
				   &\equiv \frac{\abs{V_{fi}}^2}{\hbar^2}\frac{\sin^2[\eta t/2]}{(\eta/2)^2}
\end{align*}
As \(\eta\to 0\), we interestingly see that \(P\sim t^2\). This corresponds to the first order of the sinusoidal behaviour of the time dependence. Further, we note that the probabilty as a function of \(\eta\) looks similar to a diffraction pattern.

If there is a continuum of final states, such as when an atom is ionized into a cation and a free electron, we consider the area under the peak:
\[\int_{-\infty}^\infty \frac{\sin^2(\eta t/2)}{(\eta/2)^2} = 2t\pi\]
so
\[\lim t\to\infty \frac{\sin^2(\eta t/2)}{(\eta/2)^2} = \pi t\delta(\eta/2)\]
so our transition probability becomes
\[\abs{c_f(t)}^2\to \frac{\pi t \delta(\eta/2)}{\hbar^2}\abs{V_{fi}}^2\]
as \(t\to\infty\). This is a proability per unit time. Integrating with a density of states \(\rho_f(E)\), we can find the transition rate
\begin{align*}
	\frac{\abs{c_{fi}}^2}{t}\equiv R &= \int\frac{\pi \delta(\eta/2)}{\hbar^2}\abs{V_{fi}}^2\rho_f(E)\d{E}\\
					 &=\frac{2\pi}{\hbar}\int_{-\infty}^\infty\delta(E-E_i\pm \hbar\omega)\abs{V_{fi}}^2\rho_f(E)\d{E}
\end{align*}
Simplifying, we obtain \emph{Fermi's Golden Rule} for the rate of transition:
\begin{equation}
	R = \frac{2\pi}{\hbar}\rho_f(E_f) \abs{\b{f}V_1\k{i}}^2
\end{equation}
where, of course, \(E_f = E_i\pm \hbar\omega\) is the energy of the final state.

The term \(\b{f}H_1\k{i}\) will contain within it selection rules, determined by how the perturbing hamiltonian couples different states.

\section{Electromagnetic Field}
In an electromagnetic field, we define the \emph{canonical momentum} as
\[\vb P = \vb p-\frac{q\vb A}{c}\]
where \(A\) is the vector potential. Thus, we can write the non-relativistic hamiltonian for a charged particle in an EM field as
\begin{equation}
	H = \frac{(\vb p-q\vb A/c)^2}{2m}+q\phi
\end{equation}
where \(\phi\) is the scalar potential. Expanding out the first term, we obtain
\[H = \frac{p^2}{2m}+q\phi - \frac{q}{2mc}(\vb A*\vb p+\vb p*\vb A) + \frac{q^2}{2mc^2}A^2\]
Further, recall that we have gauge transformations that leave the EM fields invariant:
\[\vb A\to\vb A+\del \chi \qquad \phi\to \phi-\frac{1}{c}\pder{\chi}{t}\]
We will use the coloumb gauge, or
\begin{equation}
	\del*\vb A = 0
\end{equation}
which fixes 
\[\del^2\phi = -4\pi\rho\]
in gaussian units.

In our semi-classical treatment, we will make \(p \to -i\hbar\del\) an operator, but leave the the vector potential \(A\) a classical vector quantitiy. Similarly, in the position representation, we will write \(V = q\phi\to q\phi(r)\). Thus, our hamiltonian in the position representation can be written
\[H\to -\frac{\hbar^2}{2m}\del^2 + V + \frac{i\hbar q}{2mc}(A*\del + \del*A) + \frac{q^2}{2mc^2}A^2
\]
Note that the term \(\del*A\) acts on \(f\) as \((\del*A)f = \del*(Af) = A*\del f + f\del*A\). Using our coulomb gauge and assuming that \(A^2\) is very small (and can be neglected) we write our hamiltonian as
\begin{equation}
	H\to -\frac{\hbar^2}{2m}\del^2 + q\phi(r) + \frac{i\hbar q}{mc}\vb A*\del
\end{equation}
so we can write our perturbing hamiltonian
\begin{equation}
	H_1 = \frac{i\hbar q}{mc}\vb A*\del
\end{equation}
Further, in the coloumb gauge, in the absence of free charges or currents, we have
\[\Box A = 0\]
so \(A\) has transverse wave solutions
\[A' = A_0 e^{i(k*r-\omega t)}\]
with
\[\del*A\sim k*A_0 = 0\]
As per usual, we consider only the real part of the wave as the physical quantity:
\[A = A' + (A')\ast\equiv A'+A''\]
We can then solve
\begin{align*}
c_f(t)&\approx -\frac{i}{\hbar}\int_0^t\d{t'}\left[e^{i(\omega_{fi}-\omega)t'}\b{f}V_1'\k{i}+e^{i(\omega_{fi}+\omega)t'}\b{f}V_1''\k{i}\right]
\end{align*}
with
\[\b{f}V_1'\k{i} =-\frac{i\hbar e}{mc}\int\d[3]{r} \ast \p_f e^{ik*r}A_0*\del \p_i\]
\[\b{f}V_1''\k{i} =-\frac{i\hbar e}{mc}\int\d[3]{r} \ast \p_f e^{ik*r}A_0\ast*\del \p_i\]
\section{Transition Rates}
Plugging back into Fermi's Golden rule, we find that we can write the transition rate in terms of a few interesting quantities. Namely, we have the time averaged Poynting vector (of a single mode) as
\[\vect{\vb P} = \frac{\omega^2}{2\pi c}\abs{\vb A_0}^2\]
and so, we have the spectral intensity
\begin{equation}
	I(\omega) = \frac{\omega^2}{2\pi c}\abs{\vb A_0^2}\rho_f(\omega)\equiv \frac{\omega^2}{2\pi c}\abs{\vb A_0 (\omega)}^2
\end{equation}
further, we can write the vector potential in terms of a polarization
\[\vb A_0 = \hat \varepsilon A_0\]
and so, we can write the rate of absorption and stimulated emission as
\begin{equation}
	R=\frac{4\pi^2 e^2}{m^2 c\omega^2}I(\omega) \begin{cases}
		\abs{\int\d[3]{r}\p\ast_fe^{i\vb k*\vb r}\hat\varepsilon *\del\p_i}^2 & \omega = +\omega_{fi}\quad\text{absorption}\\
		\abs{\int\d[3]{r}\p\ast_fe^{-i\vb k*\vb r}\hat\varepsilon\ast *\del\p_i}^2 & \omega = \omega_{fi}\quad\text{emission}\\
	\end{cases}
\end{equation}
We find that in the matrix elements, we often have \(\lambda\gg R_{\text{atom}}\) and thus, we can approximate
\begin{equation}
	e^{\pm i\vb k*\vb r}\approx 1\pm i\vb k*\vb r
\end{equation}
This is known as the \emph{electric dipole approximation}. If this rate turns out to be zero, we will obtain a selection rule. Of course, we could turn to the \emph{quadrupole transition}, but this is often only necessary in the x-ray regime. For optical spectroscopy, we need only consider this dipole approximation.

Using the fact that we can write \([\vb r,H_0] = \frac{i\hbar}{m}\vb p\)
we see
\begin{align*}
	\b f \vb p \k i&= \frac{m}{i\hbar}\b f \vb r H_0 - H_0 \vb r\k i\\
		       &=-\frac{m}{i\hbar} (E_-E_i) \b f \vb r\k r\\
		       &= im\omega_{fi}\b f\vb r\k i
\end{align*}
where in the second step, we right multiply \(\vb rH_0\k i\) and left multiply \(\b f H_0\vb r\).
Thus,
\begin{equation}
	\b f \hat\varepsilon *\del \k i = \frac{i}{\hbar}\b f \hat \varepsilon*\vb p\k i = -\frac{m\omega_{fi}}{\hbar}\b f \hat\varepsilon * \vb r \k i
\end{equation}
which is the same matrix element from when we had considered the Stark effect in hydrogen! For example, whis matrix element shows us the \(2p_z\to1s\) is allowed, while \(2p_x\to 1s\) is not. We can write more generally, that in terms of spherical harmonics,
\begin{align*}
	\hat \varepsilon * \vb r &= (\varepsilon_x \sin\theta\cos\theta + \varepsilon_y\sin\theta\sin\phi + \varepsilon_z\cos\theta)r\\
				 &=\sqrt{\frac{4\pi}{3}}\left(\varepsilon_z Y_{10}+\frac{-\varepsilon_x+i\varepsilon_y}{\sqrt{2}}Y_{11}+\frac{\varepsilon_x+i\varepsilon_y}{\sqrt{2}}Y_{1,-1}\right)r
\end{align*}
The second and third terms, in light, correspond to the circularly polarized states. If we consider these effects on a central potential,
\begin{multline*}\b{n'\ell'm'}\hat\varepsilon*\vb r\k{n\ell m}\\=\sqrt{\frac{4\pi}{3}}\int r^3\d{r}R\ast_{n'\ell'}R_{n\ell}\int\d\Omega Y\ast_{\ell'm'} \left(\varepsilon_z Y_{10}+\frac{-\varepsilon_x+i\varepsilon_y}{\sqrt{2}}Y_{11}+\frac{\varepsilon_x+i\varepsilon_y}{\sqrt{2}}Y_{1,-1}\right)Y_{\ell,m}\end{multline*}
We can evaluate these integrals in terms of Clebsch-Gordon coefficients using
\begin{equation}
	\int\d\Omega Y\ast_{\ell'm'}Y_{1 q}Y_{\ell m} = \sqrt{\frac{3}{4\pi}\frac{2\ell+1}{2\ell'+1}}C_{\ell m1q}^{\ell'm'}C_{\ell 0 1 0}^{\ell' 0}
\end{equation}
The first coefficient forces
\[m'=m+q\then \Delta m = 0,\pm 1\]
while the second forces
\[\abs{\ell-1}\leq \ell' \leq \ell+1\then \Delta\ell = 0,\pm 1\]
However, we see that \(Y_{1q}\) has odd parity, and for \(\Delta \ell =0\) we see that \(Y\ast_{\ell, m'}Y_{\ell m}\) has even parity. Thus, because \(C_{\ell010}^{\ell0} = 0\), our selection rules become
\begin{subequations}
	\begin{multicols}{2}
		\noindent \begin{equation}
			\Delta m = 0,\pm 1
		\end{equation}
		\begin{equation}
			\Delta \ell = \pm 1
		\end{equation}
	\end{multicols}
\end{subequations}
These selection rules arise even if we had quantized the electric field.

\subsection{Spontaneous Emission}
We conclude our discussion of interactions with light is spontaneous emission. Unfortunately, we cannot easily treat this in the semi-classical model, as the field is a perturbing term rather than something the system can affect. In a fully quantum treatment, the electromagnetic field is a dynamical variable, and has harmonic oscillator modes which can be excited; spontaneous emission arises from zero-point fluctuations of the field.

Rather, we will consider an argument of detailed balance following Einstein. Consider we have a system containing \(N\) atoms and a radiation energy density \(\rho = I(\omega)/c\), at an equilibrium temperature \(T\). For simplicity, we consider the atoms as a two-level system with energy separation \(\hbar\omega_0\). Because the system is at equilibrium, we expect \(N_a\) and \(N_b\), the atoms in state \(a\) and \(b\) respectively, to be constant. 

Thus, we can consider the rate of transition between these two states. Define \(A\) to be the rate of spontaneous emission, \(B_{ba}\rho(\omega_0)\) to be rate of stimulated emission, and \(B_{ab}\rho(\omega_0)\) to be the rate of absorption. We include \(\rho(\omega_i)\) as the rate of spontaneous emission and absorption to be dependent on the density of the driving radiation.

From statistical mechanics of a harmonic oscillator, we see that
\[\rho = \vect{n}\hbar\omega D(\omega)\]
Because photons are bosons (with chemical  potential zero), and using periodic boundary conditions to derive the density of states, we obtain the blackbody radiation formula
\begin{equation}
	\rho = \frac{\hbar}{\pi^2c^3}\frac{\omega^3}{e^{\beta\hbar\omega}-1}\label{eq14:bbr}
\end{equation}

At steady state, we obtain
\[\der{N_b}{t} = -N_b A-N_bB_{ba}\rho(\omega_0) + N_a B_{ab}\rho(\omega_0) = 0\]
The first two terms correspond to spontaneous and stimulated emission respectively, and is the rate that particles leave state \(b\), while the final term is absorption, and is the rate particles are promoted to state \(b\). Obviously, at equilibrium, we should find this entire rate to be zero. Solving for \(\rho\), we find
\[\rho(\omega_0) = \frac{A}{\frac{N_a}{N_b}B_{ab}-B_{ba}}\]
From statistical mechanics, we know that \(N_a/N_b\sim e^{\beta\hbar\omega_0}\). Comparing to the blackbody radiation formula~\ref{eq14:bbr}, we see that \(B_{ba}=B_{ab}\) to obtain the boson distribution:
\[\rho(\omega_0) = \frac{A/B_{ba}}{e^{\beta\hbar\omega}-1}\]
and so, we see that
\begin{equation}
	A = \frac{\omega^3\hbar}{\pi^2c^3}
\end{equation}
or, we can obtain the spontaneous emission rate from the stimulated emission rate.

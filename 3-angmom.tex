%! TEX root = 0-main.tex
\chapter{Angular Momentum}
\section{Rotations}
It is important to recall that rotations form a non-Abelian group; they do not commute. For example, for 3D rotations around axes, we have:
\[S(\phi \kh)=\begin{bmatrix}\cos\phi & -\sin\phi & 0\\\sin\phi &\cos\phi &0\\0&0&1\end{bmatrix}\quad S(\phi\ih)=\begin{bmatrix}1 & 0 & 0\\ 0& \cos\phi &-\sin\phi \\ 0 & \sin\phi&\cos\phi \end{bmatrix} \quad S(\phi \jh)=\begin{bmatrix}\cos\phi & 0 & \sin\phi \\ 0 & 1 & 0 \\ -\sin\phi & 0 & \cos\phi \end{bmatrix}\]
Note that these can be generated by knowing only \(S(\phi \kh)\); the others can be generated by cyclicly permuting the bases \(xyz \to yzx \to zxy\) and rearranging rows/columns accordingly.

In general, 
\[S(\phi \ih) S(\phi \jh) \neq S(\phi \jh) S(\phi\ih)\]
However, for small \(\Delta \phi\) such that
\[\cos\Delta\phi \approx 1-\frac{\Delta \phi^2}{2}\]
we have
\begin{equation}
	S(\Delta \phi \ih) S(\Delta \phi \jh) - S(\Delta \phi \jh) S(\Delta \phi\ih)=\begin{pmatrix}0 & -\Delta \phi^2 & 0 \\ \Delta \phi^2 & 0 & 0\\ 0&0&0\end{pmatrix}=S(\Delta \phi ^2\kh)-I
\end{equation}
Similarly, we can write this result for rotation operators:
\begin{equation}
	\left[\hat R (\Delta \phi \ih), \hat R (\Delta \phi \jh)\right]=\hat R(\Delta \phi^2\kh)-\mathbbm{1}
\end{equation}
Where the square brackets denote the commutator:
\begin{equation}
	[A,B]=AB-BA \label{eq3:commutator}
\end{equation}
Recalling the expression for the rotation operator in terms of the generator,
\[\hat R(\Delta \phi \ih)=\exp[-i\Delta\phi \hat{J}_x/\hbar]\]
and keeping only first order terms, we find
\[[\hat J_x, \hat J_y]=i\hbar\hat{J}_z\]
More generally,
\begin{equation}
	[\hat{J}_a, \hat{J}_b]=i\hbar \varepsilon_{ijk}\hat{J}_c\label{eq3:spincommutation}
\end{equation}
Note that Equation~\ref{eq3:spincommutation} applies for all anglular momenta; for example if we replace \(\hat J\) with the orbital angular momntum operator \(\hat L\)

\begin{aside}[Commuting Operators]
	Commuting operators satisfy an important and very useful characteristic: they share eigenstates. Because they share eigenstates, they can be diagonalized with the same eigenbasis. This is known as the \emph{simultaneous diagonalization theorem}. 
\end{aside}

\section{Angular Momentum \texorpdfstring{\(\hat{\vb{J}}\)}{J op}}
While this section discusses specifically the angular momentum of a spin 1 system, the results can be applied to a general angular momentum. The operator \(\hat {\vb {J} } \) is defined by
\begin{equation}
	\hat{\vb{J}} = \hat{J}_x \ih + \hat{J}_y \jh + \hat{J}_z \kh
\end{equation}
Using typical vector notation, another operator can be defined:
\begin{equation}
	\hat{J}^2\equiv \hat{\vb{J}}*\hat{\vv{J}}=\hat J_x^2+\hat J_y^2 + \hat J_z^2
\end{equation}
The commutation relation for \(\hat J^2\) wrt the basis \(J\) operators can be determined using the identity:
\begin{equation}
	[AB,C]=A[B,C]+[A,C]B
\end{equation}
We find that
\begin{equation}
	[\hat{J^2}, \hat{J}_x]=0
\end{equation}
Similar relations can be shown for each of \(\hat J_y\) and \(\hat J_z\). However, \(\hat{J}^2\) cannot be diagonalized with all of them at once. We will choose to diagonalize \(\hat{J}^2\) with \(\hat{J}_z\).

\subsection{Eigenvalues}
We define the eigenstate \(\k{\lambda, m}\) such that
\begin{subequations}
	\begin{align}
		\hat{J}^2\k{\lambda, m} &= \lambda\hbar^2\k{\lambda, m}\\
		\hat{J}_z\k{\lambda, m} &= m\hbar \k{\lambda, m}
	\end{align}
\end{subequations}

What are these eigenvalues \(\lambda\) and \(m\), and what are their values? Because \(\hat{J}_i\) are each hermitian, we know that 
\[\b{\lambda,m}\hat{J}_i^2\k{\lambda, m}= \bk{\p}{\p}\geq 0 \qquad \k\p\equiv \hat{J}_i\k{\lambda m}\]
Adding the three components together, we know that
\[\lambda \hbar^2= \b{\lambda,m}\hat J^2\k{\lambda,m}\geq 0\]

\subsubsection{Raising and Lowering Operators}
To discern other properties, we define the \emph{raising} and \emph{lowering} operators
\begin{equation}
	\hat J_\pm \equiv \hat J_x \pm i\hat J_y\label{eq3:raiselower}
\end{equation}
These two operators are adjoints of each other, and satisfy the commutation relation
\begin{equation}
	[\hat J_z, \hat J_\pm]=\pm\hbar \hat J _\pm
\end{equation}

Thus, we can evaluate
\begin{align*}
	\hat J \hat J_\pm \k{\lambda,m} &= (\hat J_\pm \hat J_z\pm \hbar \hat J_\pm)\k{\lambda, m}\\
					&= (\hat J_\pm m\hbar \pm \hbar \hat J \pm)\k{\lambda m}\\
					&= (m\pm1)\hbar\hat J_\pm \k{\lambda ,  m}\\
					&= (m\pm 1)\hbar \left(\hat J_\pm \k{\lambda, m}\right)
\end{align*}
A similar argument can be done with the \(\hat J^2\) operator to show
\begin{equation}
	\hat J^2\hat J\pm \k{\lambda, m} = \hat J_\pm \hat J^2 \k{\lambda, m}= \lambda\hbar^2\left(\hat {J}_\pm \k{\lambda, m}\right)
\end{equation}
Thus, the raising and lowering operators raise or lower the quantum number \(m\). 
\begin{equation}
	\hat J_\pm \k{\lambda,m} \propto \k{\lambda, m\pm1}
\end{equation}

\subsubsection{Spectrum of States}
Because we know that
\[\b{\lambda, m} \left(\hat J_x^2 + \hat J_y^2\right)\k{\lambda, m}\geq 0\]
we can rewrite as:
\[\b{\lambda, m}\left(\hat J^2-\hat J_z\right)\k{\lambda, m}= \hbar^2(\lambda - m^2)\geq 0\]
\begin{equation}
	\lambda\geq m^2
\end{equation}

Physically, we expect \(\lambda\) to have an upper bound; thus \(m\) must also be bounded. We define the upper bound of \(m\) to be \(j\). Thus, we restrict that:
\begin{equation}
	\hat J_+ \k{\lambda , j} = 0
\end{equation}

Then, using \(\hat J_-\hat J_+\):
\begin{equation}
	\hat J_- J_+ = \hat J^2 - \hat J_z^2 - \hbar \hat J_z
\end{equation}
we can show
\begin{align}
	\hat J_- J_+\k{\lambda, j} &=\left( \hat J^2 - \hat J_z^2 - \hbar \hat J_z\right)\k{\lambda, j}\nonumber\\
				0  &=(\lambda - j^2 - j) \hbar^2 \k{\lambda, j}\nonumber\\
				0  & = \lambda- j(j+1)\nonumber\\
				\lambda &= j(j+1)
\end{align}

Similarly, \(m\) is bounded below, by a quantity \(j'\). 
\begin{equation}
	\hat J_- \k{\lambda, j'} = 0
\end{equation}
Thus,
\begin{align}
	\hat J_+ J_-\k{\lambda, j'} &=\left( \hat J^2 - \hat J_z^2 + \hbar \hat J_z\right)\k{\lambda, j'}\nonumber\\
				0  &=(\lambda - j'^2 +' j') \hbar^2 \k{\lambda, j}\nonumber\\
				0  & = \lambda- j(j+1)\nonumber\\
				\lambda &= j'(j'+1) = j(j+1)\nonumber\\
				j'&=-j
\end{align}
While there is the additional solution of \(j'=j+1\), we do not allow this solution because it contradicts \(j\) as being the highest value of \(m\). Thus, we see that \(m\) is bounded above and below by \(\pm j\), and takes steps of 1. 
\begin{equation}
	m\in\{-j, -j+1, \ldots , j-1, j\}
\end{equation}
Note, that this restricts
\begin{equation}
	2j = n \geq 0
\end{equation}
or \(j\) is either integer of half integer. Thus, possible observables are:
\begin{subequations}
\begin{align}
	\vect{\hat{J}_z}{\hbar}\sim j &= 0,\frac{1}{2}, 1, \frac{3}{2}, \ldots\\
	\vect{\hat{J}^2}{\hbar^2}\sim \lambda = j(j+1) &= 0, \frac{3}{4}, 2, \frac{15}{4}, \ldots\\
	\frac{\sqrt{\vect{\hat{J}^2}}}{\vect{\hat{J}_z}}\sim \frac{\sqrt{j(j+1)}}{j} &= -, \sqrt{3}, \sqrt{2}, \sqrt{\frac{5}{3}},\ldots
\end{align}
\end{subequations}
Notice that the total magnitude of the angular momentum \emph{must} be greater than the \(z\) component. If they were equal, then \(J_y\) and \(J_x\) would be completely determined, which would violate the uncertainty principle (the components of the angular momentum operator do not commute).

Classically, (or perhaps in chemistry,) we view the angular momentum as a vector whose angle to the \(z\)-axis changes with \(j\). For example, for \(j=\frac{1}{2}\), the angle is \(\arccos\frac{1}{\sqrt{3}}\approx 54.7^\circ\), and for \(j=1\) the angle is \(\arccos\frac{1}{\sqrt{2}}=45^\circ\). However, in this ``vector model,'' we expect to be able to determine the other components with too much certainty; it is better to view a distribution of momentum vectors dependent on \(\theta\), and rotate it around \(\phi\) to provide a ``full'' distribution of \(\vv{J}\)'s.

\subsubsection{Uncertainty Principle}
The uncertainty relationship for two operators is related to their commutator. In particular, the Robertson Uncertainty Principle states that
\begin{equation}
	\sigma_A\sigma_B\geq \frac{1}{2}\abs{\vect{[A,B]}}\label{eq3:uncertaintyprinciple}
\end{equation}

For example, because
\[[J_x, J_y]=i\hbar J_z\]
the uncertainty relation for these operators is
\[\Delta J_x \Delta J_y \geq \frac{\hbar}{2}\abs{\vect{J_z}}\]

\section{Representations}
We want to find the constants \(c_\pm\) such that
\[\hat{J\pm} \k{j,m}=\hbar c_\pm\k{j,m\pm1}\]
Recall that 
\[\hat J_-\hat J_+ \k{j,m}= \hbar^2 [j(j+1)-m(m+1)]\k{j,m}\]
Thus
\[\b{j,m}\hat{J}_-\hat J_+\k{jm}=\hbar^2[j(j+1)-m(m-1)]\]
Note that this is equivalent to
\[(\hat J_+\k{j,m})\adj\hat{J}_+\k{j,m}=\hbar^2[j(j+1)-m(m-1)]\]
Thus, fixing the constant to be real, we get the action of the \(\hat{J}_+\) operator to be:
\begin{equation}
	\hat{J}_+\k{j,m} = \hbar\sqrt{j(j+1)-m(m+1)}\k{j,m+1}
\end{equation}
Similarly,
\begin{equation}
	\hat{J}_-\k{j,m} = \hbar\sqrt{j(j+1)-m(m-1)}\k{j,m-1}
\end{equation}
Thus, for \(j=\frac{1}{2}\), the representations are:
\begin{equation}
	\hat J_+ \xrightarrow[\hat{J}_z\text{\ rep}]{}\hbar\begin{bmatrix}0&1\\0&0\end{bmatrix}
	\qquad\qquad
	\hat J_- \xrightarrow[\hat{J}_z\text{\ rep}]{}\hbar\begin{bmatrix}0&0\\1&0\end{bmatrix}
\end{equation}
For \(j=1\) the representations are a bit more complicated. 
\begin{equation}
	\hat J_+ \xrightarrow[\hat{J}_z\text{\ rep}]{}\hbar\begin{bmatrix}0&\sqrt{2} & 0\\0&0&\sqrt{2}\\0&0&0\end{bmatrix}
	\qquad\qquad
	\hat J_- \xrightarrow[\hat{J}_z\text{\ rep}]{}\hbar\begin{bmatrix}0&0&0\\\sqrt{2}&0&0\\0&\sqrt{2}&0\end{bmatrix}
\end{equation}
Because the orientation of the problem was chosen arbitrarily, the representation for a similar operator wrt a different direction is the same:
\[\hat{J}_+^{(z)}\k{1,-1}_z = \hbar\sqrt{2}\k{1,0}_z\then\hat{J}_+^{(x)}\k{1,-1}_x = \hbar\sqrt{2}\k{1,0}_x\]

\section{Spin}
The previous discussion applies for any angular momentum. However, for spin, we often rewrite the angular momentum operator in terms of the \emph{Pauli spin vector}:
\begin{equation}
	\hat{\vb{S}} = \frac{\hbar}{2}\hat{\vv{\sigma}}
\end{equation}
Additionally, using 
\begin{equation}
	\hat S_\pm  =\hat S_x \pm i \hat S_y
\end{equation}
we see that
\begin{subequations}
	\begin{align}
		S_x & =\frac{1}{2}\left(\hat S_+  + \hat S_-\right)\\
		S_y & =\frac{1}{2i}\left(\hat S_+  - \hat S_-\right)
	\end{align}	
\end{subequations}

Thus, the representations of the Pauli spin matrices (in the z representation, and spin \(1/2\)) are given:
\begin{subequations}
	\begin{align}
		\sigma_x & = \begin{pmatrix}
			0&1\\1&0
		\end{pmatrix}\\
		\sigma_y & = \begin{pmatrix}
			0 & -i \\ i & 0
		\end{pmatrix}\\
		\sigma_z & = \begin{pmatrix}
			1&0\\0&-1
		\end{pmatrix}
	\end{align}
\end{subequations}

Similarly, for spin 1, using the representation of the raising and lowering operators, we get:
\begin{subequations}
	\begin{align}
		\hat S_z &= \frac{\hbar}{\sqrt{2}} \begin{pmatrix}
			0 & 1 & 0 \\
			1 & 0 & 1 \\
			0 & 1 & 0 
		\end{pmatrix}\\
		\hat S_y & = \frac{\hbar}{\sqrt{2}} \begin{pmatrix}
			0 & -i & 0 \\
			i & 0 & -i \\
			0 & i & 0 
		\end{pmatrix}\\
		\hat S_z & = \hbar \begin{pmatrix}
			1 & 0 & 0\\
			0 & 0 & 0\\
			0 & 0 & -1
		\end{pmatrix}\\
		\hat S^2 & = 2\hbar^2 \begin{pmatrix}
			1 & 0 & 0\\
			0 & 1 & 0\\
			0 & 0 & 1
		\end{pmatrix}
	\end{align}
\end{subequations}
It can be verified through routine calculation that these matrices satisfy the angular momentum commutation relations.

\begin{aside}[Stern-Gerlach revisited]
	Let us follow a SGy1 apparatus by a SGz apparatus, in the spin 1 case. What are the probabilities of each output of the SGz apparatus?

	To get the answer, we need to write the \(\k{1,1}_y\) state in terms of the \(z\) basis. We can use the representation of the \(\hat S_y\) operator to solve for eigenvalues and eigenvectors. We know already the eigenvalues of the problem as the observable values of the spin. If we did not know the eigenvalues, we find them for an arbitrary linear operator \(\hat{O}\) using the characteristic equation:
	\begin{equation}
		\det\left[\hat{O}-\lambda \mathbbm{1}\right]=0
	\end{equation}
	Substituting the eigenvalues into the eigenvalue equation:
	\[\hat S_y \k{1,m}= m\hbar \k{1,m}\]
	We get the following representation for \(\k{1,1}_y\):
	\[\k{1,1}_y\sim\frac{1}{2} \begin{pmatrix}
		1 \\ i\sqrt{2} \\ -1
	\end{pmatrix}\sim\frac{1}{2}\k{1,1}_z + \frac{i}{\sqrt{2}}\k{1,0}_z -\frac{1}{2}\k{0,-1}\]
	Using these coefficients, we can easily find the outputs of the SGz apparatus.
\end{aside}

\subsection{Spin Direction}
We can generate a spin operator for an arbitrary direction \(\vv{n}\) using
\begin{equation}
	\hat S_n = \hat{\vb{S}} * \vv{n}
\end{equation}

For the pauli spin vector,
\[\sigma = \sigma_x \e x + \sigma_y \e y + \sigma_z \e z\]
and a direction restricted to the \(xy\) plane, 
\[\vv{n}=\cos\phi\e x + \sin\phi \e y\]
we can multiply and simplify to be:
\[\vv{S}_n = \frac{\hbar}{2} \begin{pmatrix}
	0 & e^{-i\phi}\\e^{i\phi} & 0
\end{pmatrix}\]
Solving for eigenvalues. 
\[\det(\vv{S}_n-\mu I) = 0 = \mu^2-1\]
and eigenvector of \(\mu=1\), we see that 
\[\abs{\b{+z}{+n}}^2=\frac{1}{2}\]
Because only relative phase matters, we can fix
\[\bk{+z}{+n}=\frac{1}{\sqrt{2}}\]
This forces
\[\bk{-z}{+n}=\frac{e^{i\phi}}{\sqrt{2}}\]
Thus, the eigenvector is
\[\k{+n}=\frac{1}{\sqrt{2}}\k{+z}+\frac{e^{i\phi}}{\sqrt{2}}\k{-z}\]
Similarly, solving for \(\mu=-1\)
\[\k{-n}=\frac{1}{\sqrt{2}}\k{+z}-\frac{e^{i\phi}}{\sqrt{2}}\k{-1}\]

A similar proof for an arbitrary direction in 3D space can be found in HW4.

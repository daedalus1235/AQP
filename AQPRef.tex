\documentclass{article}
\usepackage{hw}
\usepackage{multicol}
\geometry{paperheight=8.5in,
	paperwidth=5.5in,
	heightrounded,
	margin=0.5in
}
\hfuzz=0.5in

\title{Reference for AQP}
\author{\vspace{-1em}Charles Yang}
\date{\vspace{-1em}Last Updated \today}

\begin{document}
\vspace{-1in}
\maketitle
\vspace{-.5in}
\section{Dirac Notation}
\subsection{Representations}
\subsubsection{Position}
The position space representation of the momentum operator is
\begin{equation}
	\vb p\sim -i\hbar \del
\end{equation}

\section{Angular Momentum}
The angular momentum operator \(\vb J\) has two corresponding eigenvalues:
\begin{subequations}
	\begin{multicols}{2}
		\noindent \begin{equation}
			J^2\k{j,m} = j(j+1)\hbar^2
		\end{equation}
		\begin{equation}
			J_z\k{j,m} = m\hbar
		\end{equation}
	\end{multicols}
\end{subequations}
The \(J_z\) operator has two associated raising and lowering operators, which obey
\begin{equation}
	J_\pm\k{j,m} = \hbar\sqrt{j(j+1)-m(m\pm 1)}\k{j,m\pm 1}
\end{equation}
We can write the other components of the angular momentum operator as
\begin{subequations}
	\begin{multicols}{2}
		\noindent \begin{equation}
			J_x = \frac{J_++J_-}{2}
		\end{equation}
		\begin{equation}
			J_y = \frac{J_+-J_-}{2i}
		\end{equation}
	\end{multicols}
\end{subequations}
The following commutation relations are obeyed,
\begin{subequations}
	\begin{multicols}{2}
		\noindent \begin{equation}
			[J_a,J_b] = i\hbar\varepsilon_{abc}J_c
		\end{equation}
		\begin{equation}
			[J_z,J_\pm] = \pm\hbar J_\pm
		\end{equation}
	\end{multicols}
\end{subequations}
Due to linearity, when we add two angular momenta, \(\vb J=\vb J_1=\vb J_2\) all components are additive. In particular,
\begin{equation}
	J_{\pm} = J_{1\pm}+J_{2\pm}
\end{equation}
However, \(\vb J^2 = \vb J_1^2+\vb J_2^2+2\vb J_1*\vb J_2\). We can expand the last term in a more useful way as 
\begin{equation}
	2\vb J_1*\vb J_2 = J_{1+}J_{2-}+J_{1-}J_{2+}+2J_{1z}J_{2z}
\end{equation}
We can transform from the basis from \(\k{j_1,m_2,j_2,m_2}\) to \(\k{j,m,j_1,j_2}\) by using 
\begin{subequations}
	\begin{multicols}{2}
		\noindent \begin{equation}
			m= m_1+m_2
		\end{equation}
		\begin{equation}
			\abs{j_1-j_2}\leq j \leq j_1+j_2
		\end{equation}
	\end{multicols}
\end{subequations}
We can use our overall \(J_\pm\) operator to generate other linear combinations.
\subsection{Clebsch-Gordon Coefficients}
\section{Symmetry}
If two operators commute, then they can be diagonalized wrt the same set of eigenvectors.
\section{Central Potentials}
For a spherical wavefunction, the angular dependence is given entirely in angular momentum:
\begin{equation}
\bk{\theta,\phi}{\ell, m} = Y_{\ell,m}(\theta,\phi)
\end{equation}
We can find the \emph{spherical harmonics} by using the Legendre polynomials
\begin{equation}Y_{\ell,0} = \sqrt{\frac{2\ell+1}{4\pi}}P_\ell(\cos\theta)\end{equation}
which have the orthonormality constraint
and applying the raising/lowering operators
\begin{equation}
L_\pm\simeq \frac{\hbar}{i}e^{\pm i\phi}\left(\pm i\pder{}{\theta}-\cot\theta\pder{}{\phi}\right)
\end{equation}
The spherical harmonics obey the orthonormality condition
\begin{equation}
	\int\d\Omega Y\ast_{\ell, m} Y_{\ell', m'} = \delta_{\ell,\ell'}\delta_{m,m'}
\end{equation}
and have parity given
\begin{equation}
\Pi Y_{\ell m} = (-1)^\ell Y_{\ell m}
\end{equation}
In general, for \(R = u/r\), the radial Schr\"odinger equation is
\begin{equation}
	\left[-\frac{\hbar^2}{2\mu}\der{}{r2}+\frac{\ell(\ell+1)\hbar^2}{2\mu r^2}+V(r)\right]u = Eu
\end{equation}
which can be solved using power-series solutions. The existence of the centrifugal term, \(\frac{\ell(\ell+1)\hbar^2}{2\mu r^2}\), forces \(R_{\ell\neq0}(r=0) = 0\). 

\subsection{Hydrogen-like Atom}
Along with the radial wavefunction, the orbitals of hydrogen-like atoms are denoted by the state \(\k{n,\ell,m}\). These orbitals have the property that the number of angular nodes is equal to \(\ell\), the number of radial nodes \(n_r\), and that \(n = n_r+\ell+1\). 

Properties of the hydrogen-like atom are
\begin{subequations}
	\begin{multicols}{2}
		\noindent \begin{equation}
			E_n = -\frac{\mu c^2 Z^2\alpha^2}{2n^2}
		\end{equation}
		\begin{equation}
			a_0 = \frac{\hbar c}{\alpha \mu c^2}
		\end{equation}
	\end{multicols}
	We will often write the wavefunction in terms of the quantities
	\begin{multicols}{2}		
		\noindent \begin{equation}
			\rho = \sqrt{\frac{8\mu\abs{E}}{\hbar^2}}=\frac{2Z}{n}\frac{r}{a_0}
		\end{equation}\begin{equation}
			\alpha = \frac{e^2}{\hbar c}
		\end{equation}
	\end{multicols}
\end{subequations}
\section{Special Systems}
\subsection{Harmonic Oscillator}
The harmonic oscillator has raising and lowering operators defined
\begin{subequations}
	\begin{multicols}{2}
		\noindent \begin{equation}
			a\k{n} = \sqrt{n}\k{n-1}
		\end{equation} \begin{equation}
		a\adj\k{n} = \sqrt{n+1}\k{n+1}
		\end{equation}
	\end{multicols}
	Further, we can write position and momentum in terms of them
	\begin{multicols}{2}
		\noindent \begin{equation}
			x = \sqrt{\frac{\hbar}{2m\omega}}(a+a\adj)
		\end{equation} \begin{equation}
		p = -i\sqrt{\frac{m\omega\hbar}{2}}(a-\adj a)
		\end{equation}
	\end{multicols}
\end{subequations}
The 3D harmonic oscillator in spherical coordinates has energy
\begin{equation}
	E_n = (2n_r+\ell+3/2)\hbar\omega = (n+3/2)\hbar\omega
\end{equation}
The degeneracy of each level can be determined using the relation \(n = 2n_r+\ell\).


\section{Perturbation}
Consider a perturbing hamiltonian \(H_1\) added to a known hamiltonian \(H_0\). We can approximate the energies and eigenstates for the hamiltonian given \(H=H_0+H_1\) using perturbation theory.
\subsection{Non-degenerate Perturbation}
The first order energy correction is given in terms of the unperturbed wavefunctions.
\begin{equation}
E_n^{(1)} = \b{\f_n^{(0)}}H_1\k{\f_n^{(0)}}\end{equation}
The first order corrections to the wavefunction may be obtained as
\begin{equation}\bk{\f_k^{(0)}}{\f_n^{(1)}} = \frac{\b{\f_k^{(0)}}H_1\k{\f_n^{(0)}}}{E_n^{(0)}-E_k^{(0)}}\end{equation}
so our new eigenstate can be written
\begin{equation}
	\k{\p_n} \approx \k{\f_n^{(0)}}+\sum_{k\neq n} \k{\f_k^{(0)}}\frac{\b{\f_k^{(0)}}H_1\k{\f_n^{(0)}}}{E_n^{(0)}-E_k^{(0)}}\end{equation}
A second order correction to the energy may be found
\begin{equation}
	E_n^{(2)} = \b{\f_n^{(0)}}H_1\k{\f_n^{(1)}} = \sum_{k\neq n}\frac{\abs{\b{\f_k^{(0)}}H_1\k{\f_n^{(0)}}}^2}{E_n^{(0)}-E_k^{(0)}}
\end{equation}
\subsection{Degenerate Perturbation}
We see that when there are degenerate eigenspaces, that our non-degenerate perturbations fail. Instead, we \emph{restrict the perturbing hamiltonian} \(H_1\) to act only on the degenerate eigenspace \(E_n\) in question, and diagonalize the restricted hamiltonian \(H_1'\). Thus,
\begin{enumerate}
	\item The first-order energy corrections will be the eigenvalues of the diagonalized \(H_1'\).
	\item The new eigenstates will be the linear combination of unperturbed degenerate eigenstates which diagonalize \(H_1'\).
\end{enumerate}
Often, we can make use of symmetries and commutation relations to force large regions of the restricted hamiltonian to vanish.
\subsection{Time-Dependent Perturbation}
Following a similar procedure to the non-degenerate perturbation, but for the time-dependent Schr\"odinger equation, we obtain the time-dependent coefficients to first order as
\begin{equation}
	c_f(t)=\delta_{fi} - \frac{i}{\hbar}\int_0^t\d{t'} e^{i(E_f^{(0)}-E_i^{(0)})t/\hbar}\b{E_f^{(0)}} H_1\k{E_i^{(0)}}
\end{equation}
The probability of a transition is correspondingly 
\begin{equation}
	P_f(t) = \abs{c_f(t)}^2
\end{equation}
From this transition probability, we can obtain Fermi's Golden Rule for the transition rate in terms of a density of states \(\rho(E)\)
\begin{equation}
	R = \frac{2\pi}{\hbar}\rho_f(E_f)\abs{\b{f}V_1\k{i}}^2
\end{equation}

\subsection{Variational Method}
The ground state of a hamiltonian is the lowest energy state. Thus, we can take a guess a trial wavefunction with a free parameter, vary that parameter, and take the minimum energy as an upperbound for the ground state energy.


\section{Indistinguishable Particles}
Because electrons are fermions, the total wavefunction must be antisymmetric. Thus, either the spatial wavefunction or the spin wavefunction is symmetric and the other antisymmetric. The symmetric spin wavefunction is a \emph{triplet state} corresponding to \(\k{1,m}\), while the antisymmetric spin wavefnuction is a \emph{singlet state} corresponding to \(\k{0,0}\). Spin eigenstates can be factored into a position wavefunction and a spin wavefunction.

The indistinguishability of particles leads to an ``exchange interaction'' which can cause a shift in the energy \(K\).

\section{Scattering}
When we send a planewave \(\p_{inc} = Ae^{ikz}\) toward a central scattering potential, we see that the far limit of the scattered wavefunction can be written
\begin{equation}
	\p_{sc} = Af(\theta,\phi)e^{ikr}/r
\end{equation}
The probability current can be found
\begin{equation}
	\vb j =\frac{\hbar}{2\mu i}(\p\ast\del\p-\p\del\p\ast)
\end{equation}
Dividing the incident probability flux by the scattered probability flux, we find the \emph{differential cross section} as a function of the \emph{scattering amplitude}.
\begin{equation}
	\der\sigma\Omega = \abs f^2
\end{equation}
For central potentials, we will have cylindrical symmetry which allows us to discard \(\phi\) dependence.
\subsection{Born Approximation}
When the incident energy is large compared to the potential (or, roughly, the scattering intensity is small compared to the incoming intensity), we can find the scattering amplitude as a fourier transform
\begin{equation}
	f(\theta,\phi) = -\frac{\mu}{2\pi\hbar^2}\int\d[3]{r'}V(\vb r')e^{i\vb q*\vb r'}
\end{equation}
where \(\vb q = \vb k_i-\vb k_f\) gives how much the momentum of the particle changes. Often, as \(\vb k_i\sim\vb k_f \equiv \vb k\), we can write 
\begin{equation}\vb q \sim 2k\sin(\theta/2)\end{equation}

In particular, for the Yukawa potential (whose special cases include the coulomb potential)
\begin{equation}
	V(r) = ge^{-m_0 r}/r
\end{equation}
we find
\begin{equation}
	f_{\text{Yukawa}}(\theta) = \frac{-2\mu g}{\hbar^2(m_0^2+q^2)}
\end{equation}

\subsection{Partial Wave Expansion}
In the partial wave expansion, we use the completeness of the Legendre polynomials (and cylindrical symmetry) to write
\begin{equation}
	f(\theta) = \sum_\ell (2\ell+1)\frac{e^{i\delta_\ell}}{k}\sin\delta_\ell P_\ell(\cos\theta)
\end{equation}
From this, we can separate our analysis into each of the different values of \(\ell\). Typically, we analyze low energy waves, which causes large \(\ell\) to be negligible.

Integrating, we find we can write the \(\ell\)\textsuperscript{th} partial subsection as
\begin{equation}
	\sigma_\ell = \frac{4\pi}{k^2}(2\ell+1)\sin^2\delta_\ell
\end{equation}

We can determine this phase shift by analyzing the asymtotic expressions for the wavefunction as \(r\to\infty\) with and without the potential.

Often, analysis with \(u = rR\) is easier, because rather than bessel functions, we need only consider trig functions.

\subsection{Resonance}
When an incoming particle has energy near (but above) a bound state, the potential tries to pull it into the bound state, which causes it to get stuck briefly. This may result, for example, from the centrifugal barrier altering a finite well.

In this case, to leading order, we have
\begin{equation}
	\sigma_\ell\approx \frac{4\pi}{k^2}(2\ell+1)\frac{\Gamma^2/4}{(E-E_0)^2+\Gamma^2/4}\qquad \at{\der{\delta_\ell}{E}}{E_0} = \frac{2}{\Gamma}
\end{equation}
Thus, we see a resonance peak near \(E = E_0\), whose thickness (FWHM) is related to the lifetime of the quasi-bound state.

\section{Light}
If we plug in the canonical momentum to the hamiltonian, we obtain the perturbing hamiltonian
\begin{equation}
	H_1 = \frac{i\hbar q}{mc}\vb A*\del
\end{equation}
If we plug this into Fermi's golden rule with a complex valued \(\vb A_0\) amplitude of the vector potential waves, we obtain that the rate of transitions depends on
\begin{multline}
	\b{n'\ell'm'}\hat n*\vb r\k{n\ell m}\\
	=\sqrt{\frac{4\pi}{3}}\int r^3\d{r} R_{n'\ell'}\ast R_{n\ell}\\
	\times\int\d\Omega Y_{\ell'm'}\ast\left(n_zY_{10}+\frac{-n_x+in_y}{\sqrt{2}}Y_{11}+\frac{n_x+in_x}{\sqrt{2}}Y_{1,-1}\right)Y_{\ell m}
\end{multline}
or, in terms of Clebsch-Gordon coefficients
\begin{equation}
	\int\d\Omega Y_{\ell'm'}\ast Y_{1q}Y_{\ell m} = \sqrt{\frac{3}{4\pi}\frac{2\ell+1}{2\ell'+1}}C_{\ell m 1 q}^{\ell'm'} C_{\ell 0 1 0}^{\ell'0}
\end{equation}
which gives the selection rules
\begin{subequations}
	\begin{multicols}{2}
		\noindent \begin{equation}
			\Delta m = 0,\pm 1
		\end{equation}
		\begin{equation}
			\Delta\ell = \pm 1
		\end{equation}
	\end{multicols}
\end{subequations}
Using a black-box cavity at thermal equilibrium, we can use a steady-state rate law to show that we can relate the rate of spontaneous emission \(A\) to the rates of stimulated emission \(B_{ba}\rho(\omega_0)\) and absorption \(B_{ab}\rho(\omega_0)\) via
\begin{subequations}
	\begin{multicols}{2}
		\noindent \begin{equation}
			B_{ab} = B_{ba}
		\end{equation}
		\begin{equation}
			A = \frac{\omega^3\hbar}{\pi^2c^3}B_{ba}
		\end{equation}
	\end{multicols}
\end{subequations}


\section{Math}
The first three legendre polynomials are given
\begin{equation}
	P_0 = 1 \qquad\quad P_1 = x \qquad\quad P_2 =\frac{3}{2}x^2-\frac{1}{2} \quad\qquad P_3 = \frac{5}{2}x^3-\frac{3}{2}x
\end{equation}
A useful trig identiy is
\begin{equation}
	A\sin(x)+B\cos(x) = \sqrt{a^2+b^2}\sin(\theta+\delta)\qquad \tan\delta = B/A
\end{equation}
\subsection{Integrals}
Recall the Gamma function is define
\begin{equation}
	\Gamma(z) = \int_0^\infty t^{z-1}e^{-t}\d{t}
\end{equation}
It satisfies  the recurrence relation
\begin{equation}
	\Gamma(z+1) = z\Gamma(z)
\end{equation}
with particular values 
\begin{subequations}
	\begin{multicols}{2}
		\noindent \begin{equation}
			\Gamma(n+1)=n!
		\end{equation}
		\begin{equation}
		\Gamma(1/2) = \sqrt{\pi}
		\end{equation}
	\end{multicols}
\end{subequations}
The gaussian integral, for \(\Re(a)>0\) and \(a\neq 0\) is given
\begin{equation}
	\int_{-\infty}^\infty\d{x}e^{-ax^2+bx} = \sqrt{\frac{\pi}{a}}e^{b^2/4a}
\end{equation}
and a generally useful integral is
\begin{equation}
	\int_0^\infty\d{t'}e^{i\omega t'}e^{-t'/\tau} = \frac{\tau}{1-i\omega \tau}
\end{equation}
\end{document}

%! TEX root = 0-main.tex
\chapter{Path Integral Formalism}
A na\"\i ve interpretation of the wavefunction is to state that the wavefunction \emph{is} the particle; however, this results in the uncomfortable concept of ``wavefunction collapse'' upon measurement. Instead, the Path Integral method considers the wavefunction as the possible trajectories, or \emph{paths} a particle can take.

In analogy, we can consider Huygen's principle of the decomposition of wavefronts as a set of point emitters for the propagation of an EM wave. Similarly, if we evolve a point point particle, at every point along its trajectory, its velocity is arbitrary, albeit with an accumulation of phases. Summing over every path and accounting for self-interference (resulting from phase) we obtain a set of paths that dominate the classical limit (\(\hbar\to 0\)); these paths satisfy the principle of least action, and thus are called the \emph{classical trajectory}, as opposed to the quantum paths. Feynmann showed---in his Ph.D. thesis---that the laws of quantum mechanics arise naturally from summing these paths.

\section{Transition Amplitude}
We consider the transition amplitude of a system evolving \(\k{x_0,t_0}\to\k{x',t'}\) as
\begin{align}
	\bk{x',t'}{x_0,t_0}&=\b{x'}\hat U(t'-t_0)\k{x_0}\nonumber\\
			   &=\b{x'}e^{-i\hat H(t'-t_0)/\hbar}\k{x_0}\label{eq8:propagator}
\end{align}
From this, we can then find the wavefunction
\begin{align*}
	\p(x',t') &= \bk{x'}{\p(t')}\\
		  &=\b{x'}e^{-i\hat H(t'-t_0)/\hbar}\k{\p(t_0)}\\
		  &=\int_{-\infty}^\infty\d{x_0}\bk{x',t'}{x_0,t_0}\bk{x_0}{\p(t_0)}
\end{align*}
Thus, the quantity shown in Equation~\ref{eq8:propagator} is known as the propagator; it is a type of kernel in a Green's function. This propagator is considered to be starting at every possible starting point \(x_0\) and following every possible path.

For a free particle, the propagator become
\begin{align}
	\bk{x',t'}{x_0,t_0}&=\b{x'}e^{-i\frac{\hat p_x^2}{2m}(t'-t_0)/\hbar}\k{x_0}\nonumber\\
			   &=\int_{-\infty}^\infty\d{p}\b{x'}e^{-i\frac{\hat p_x^2}{2m}(t'-t_0)/\hbar}\k{p}\bk{p}{x_0}\nonumber\\
			   &=\int_{-\infty}^\infty\d{p}e^{-i\frac{p^2}{2m}(t'-t_0)/\hbar}\bk{x'}{p}\bk{p}{x_0}\nonumber\\
			   &=\frac{1}{2\pi\hbar}\int_{-\infty}^\infty\d p e^{ip(x'-x_0)}e^{-ip^2(t'-t_0)/\hbar}\nonumber\\
			   &=\sqrt{\frac{m}{2\pi \hbar i (t'-t_0)}}\exp\left[i\frac{m(x'-x_0)^2}{2\hbar(t'-t_0)}\right]\label{eq8:freeprop}
\end{align}
Equation~\ref{eq8:freeprop} is one of the most useful results in this chapter, and can be applied to a variety of problems. Examining the form of the propagator, for fixed \(t\equiv t'-t_0\), we see the form is something like \(\exp[iAx^2]\)-a wave whose period decreases as \(x\equiv x'-x_0\) goes to infinity. For large \(x\gg\lambda\), we can solve for the wavelength as
\[2\pi = \frac{m(x+\lambda)^2}{2\hbar t}-\frac{mx^2}{2\hbar t}\]
This yields
\begin{equation}
	\lambda = \frac{2\pi\hbar}{m(x/t)}\sim \frac{h}{p}
\end{equation}
A similar result can be shown for the time dependence\footnote{This is done in HW}. The propagator can be also be interpreted as the evolution of a delta function, or of the evolution of a single eigenstate an eigenstate decomposition.

\section{Paths to Propagators}
We can also compute a propagator in terms of \emph{paths}. For small time differences \(\Delta t \equiv t'-t_0\), we can write
\[e^{-i\hat H \Delta t/\hbar}\approx 1 - \frac{i}{\hbar}\hat H\Delta t\]
Using this approximation,
\begin{align*}
	\b{x'}e^{-iH\Delta t/\hbar}\k{x}
	&\approx \b{x'}1-\frac{i}{\hbar}\hat H\Delta t \k{x}\\
	&=\int \d{p}\bk{x'}{p}\b{p}1-\frac{i}{\hbar} \left(\frac{p^2}{2m}+V(x)\right)\delta T\k{x}\\
	&\approx \int\bk{x'}{p}\bk{p}{x}e^{0iE\Delta t/\hbar}\\
	&=\int_{-\infty}^\infty\d{p}\exp\left[\frac{i}{\hbar}\left(p\frac{(x'-x)}{\Delta t}-E(p,x)\right)\Delta t\right]\\
	&=\sqrt{\frac{m}{2\pi \hbar i\Delta t}}\exp\left[\frac{i}{\hbar}*\frac{m\Delta t}{2}\left(\frac{x'-x_0}{\Delta t}\right)^2-\frac{i}{\hbar}V(x_0)\Delta t\right]
\end{align*}

To consider a finite travel time \(t\), we use \(N\) small steps, such that \(N\Delta t = t'-t_0\). Thus, 
\begin{align*}
\bk{x',t'}{x_0,t_0} 
&= \b{x'}e^{-i\hat H \Delta t/\hbar}e^{-i\hat H\Delta t/\hbar}\dots e^{-i\hat H\Delta t/\hbar}\k{x_0}
\intertext{To evaluate this expression, we repeatedly apply the completness relation:}
&= \idotsint\d[N-1]{x}\b{x'}e^{-i\hat H \Delta t/\hbar}\k{x_{N-1}}\b{x_{N-1}}\dots\k{x_1}\b{x_1} e^{-i\hat H\Delta t/\hbar}\k{x_0}
\intertext{The intermediate values of \(\k{x_i}\) define the arbitrary paths that the particle travels. These paths need not be smooth, or continuous, but fill space-time uniformly. The integral is carried out over all \(i\neq0\), or all pahts. Because each integral is independent, we can substitute the previously obtained result. Thus,}
&=\lim_{N\to\infty}\idotsint\d[N-1]{x}\left(\frac{m}{2\pi\hbar i \Delta t}\right)^{N/2}\exp\left\{\frac{i}{\hbar}\sum_{i=1}^\N \left[\frac{m}{2}\left(\frac{x_i-x_{i-1}}{\Delta t}\right)^2-V(x_{i-1})\right]\right\}\\
&=\idotsint\d[N-1]{x}\left(\frac{m}{2\pi\hbar i\Delta t}\right)^{N/2} e^{iS/\hbar}
\end{align*}
where \(S[x(t)] = \int_{t_0}^{t'} L(x,\dot x)\d{t}\) is the action. This is often rewritten in functional notation as
\begin{equation}
	\bk{x',t'}{x_0,t_0} = \int_{x_0}^{x'}\mathscr{D}[x(t)]e^{-S[x(t)]/\hbar}
\end{equation}
where
\[\int_{x_0}^{x'}\mathscr{D}[x(t)] = \idotsint\d[N-1]{x}\left(\frac{m}{2\pi\hbar i \Delta t}\right)^{N/2}\]

\subsection{Classical Limit}
Consider the wavefunction
\[\p(x',t') =\int_{-\infty}^\infty\d{x_0}\bk{x',t'}{x_0,t_0}\p(x_0,t_0)\]
The propagator takes the particle from \((x_0,t_0)\) to \((x',t')\). Classically, we expect the particle to take only a single path, but this formalism claims that all possible paths contribute to the wavefunction.

Classically, paths are large, so \(S/\hbar\gg1\). Thus, \(e^{iS/\hbar}\) oscillates rapidly between pahts. Further, we know that for a classical path, we expect it to follow the principle of least action
\[\at{\delta S}{x(t)=  x_{cl}(t)} = 0\]

\begin{aside}[Neutrons]
	Consider a free neutron with wavelength \(\lambda = \SI{1.419}{\angstrom}\) travelling over a distance \(d = \SI{10}{cm}\). The action is given 
	\[S = \left(\frac{1}{2}mv^2\right)t = \frac{1}{2}mv^2\frac{d}{v} =\frac{dp}{2}= \frac{d\hbar k}{2}=\SI{2e9}{\hbar}\]
	Although this seems like a large action, other paths have \emph{even larger} actions.
\end{aside}

For large \(S\), we claim that paths that significantly deviate from the classical path \(x_{cl}(t)\) make zero contribution to the integral. Classical paths and nearby paths dominate by method of stationary phase. That is, only paths with action near the minimum contribute significantly to the trajectory. 

\begin{aside}[Least Action Paths]
	Consider a free particle in 1D. The lagrangian is given \(L = \frac{1}{2}mv^2\), and so the action is \(S = \frac{1}{2}m\int_0^{t'}v^2\d{t}\). We claim that \(v = \frac{x'}{t'}\) gives the minimum action.The action of this path is given \(\frac{1}{2}mv^2t'\). Considering instead a piecewise velocity of \(2v\) for \(t'/4\) and \(\frac{2}{3}v\) for \(\frac{3}{4}t'\), the action instead becomes \(S = \frac{1}{2}mv^2t'(1+\frac{1}{3})\), which is indeed larger.	
\end{aside}

\section{Examples}
\subsection{Problem 8.6}
A beam of particles with charge \(q\) is split and shone through two tubes, with potentials \(V_1,V_2\) respectively. After exititng the tubes, the particles go into a detector. Within the tubes, the potential is given \(U_i = q\phi_i\). The particles then experience an additional phase factor due to the potential as, for instace,
\[\psi_1(x,t) = \psi_{1}^{(0)}(x,t)\exp\left[-\frac{iq}{\hbar}\int_{t_0}^{t'}\phi_1\d{t}\right] = \psi_{1}^{(0)}e^{-\frac{iq}{\hbar}V_1(t'-t_0)}\]
where \(\psi_1^{(0)}\) is a path in the absense of the applied potential. This leads to a phase difference at detection due to the difference in the two voltages. Let \(\psi_{1}^{(0)} = \psi_2^{(0)} = \psi^{(0)}\). Then, the detected wavefunction is
\[\psi_1+\psi_2 = \psi^{(0)}e^{-\frac{iq}{\hbar}V_1 (t'-t_0)}\left(1+e^{-\frac{iq}{\hbar}(V_2-V_1)(t'-t_0)}\right)\]
The relative phase difference due to the voltage difference causes interference.

\subsection{Neutrons in Gravitational Potential}
A monoenergetic beam of neutrons is split using crystal diffraction into two beams, then are recombined, forming a parallelogram. The ``vertical distance'' between the two beams is called \(\ell_2\), and the length of the horizontal segments are \(\ell_1\). The entire apparatus can be rotated to an angle \(\delta\), as to change the gravitational potential experienced by the different paths (an apparatus parallel to the ground  experiences a constant gravitational potential, while one perpendicular does not.) 

In the ``higher'' path, the horizontal segment has a relative potential of \(mg\ell_2\sin\delta\) compared to the horizontal on the lower path. The resultant phase difference is given
\[\Delta \phi = \Delta V \Delta t /\hbar= mg\ell_2\sin\delta * \frac{\ell_1}{v} * \frac{1}{\hbar}\]
Indeed, experiment showed that there was a phase difference when \(\delta\) was such that the apparatus was not horizontal. 

Note, however, that this neglects the velocity change in the beam as a result of the change in gravitational potential. While the two diagonal arms cancel each other out, the horizontal paths consequently have a velocity difference. We expect this to have a phase difference in addition to that caused by action due to a difference in the time it takes to travel along each path. 
